\newpage
\thispagestyle{empty}

\begin{center}
\textbf{DISSERTATION ABSTRACT}
\end{center}

\begin{enumerate}[leftmargin=0em, itemindent=1.8em]
\item Title: \begin{minipage}[t]{0.75\textwidth} \vspace{-1.9em}\singlespacing{EXPLORING THE CONCEPT OF YOMI AS AN ARTIFICIAL \\ INTELLIGENCE FRAMEWORK: \MakeUppercase{Yomi AI Framework in RoShamBo}}\vspace{0.7em}
\end{minipage}

%\begin{enumerate}[leftmargin=3.2em,label*=\arabic*.]
\begin{enumerate}[leftmargin=1.5em,label*=\arabic*.]
% Total No of Pages -> Last Page
% Text no of Pages-> From chap 1 to chapt 5 w/o tables and figures

\item Total No. of Pages: \ifnotUCFormat  \else 212 \fi	%\ref{totPages}
\item Text No. of Pages: \ifnotUCFormat   \else 186 %-26 figure pages  %\pageref{TotPages}
							\fi

\end{enumerate}


\item Author: \MakeUppercase{\FullNameFamilyFirst}

\item Type of Document: Dissertation

\item Type of Publication: Unpublished

\item Accrediting Institution: \begin{minipage}[t]{0.5\textwidth} \vspace{-1.9em}\singlespacing{University of the Cordilleras   \\
Gov. Pack Road, Baguio City     \\
CHED-CAR}\vspace{0.5em}
\end{minipage}

\item Sponsor: CHED-FDP

\item Key words: \begin{minipage}[t]{0.7\textwidth} \vspace{-1.9em}\singlespacing{\Keywords}\vspace{0.5em}\end{minipage}

\item Abstracts:

\singlespacing{

\begin{enumerate}[label*=\arabic*., leftmargin=0em,  listparindent=3em]

\item \begin{center}\textbf{Rationale / Background of the Study} \end{center}

\ifnotUCFormat
\else
	\hspace{1em}
\fi

The Yomi AI framework is inspired after a concept refined by the competitive fighting games community. Yomi is a Japanese word meaning: \textit{to read}. In the context of competitive games, this refers to the idea of reading an opponent's state of mind. The Yomi concept is useful when there's a need to anticipate an opponent's strategy or move. This AI was modeled from the dissertation author's personal experience at participating in competitive games and his own personal application of the Yomi concept.

\ifnotUCFormat
\else
	\hspace{1em}
\fi

\item \begin{center}\textbf{Summary}\end{center}

\ifnotUCFormat
\else
	\hspace{1em}
\fi

% Note: Summary needs to be single-spaced. If your latex style uses double-space, use the code in 
%       https://www.physicsforums.com/threads/switch-from-double-to-single-spacing-in-latex.325863 or
%       http://tex.stackexchange.com/questions/48741/temporarily-increase-line-spacing
%       to set the summary to single space

The researcher presents a novel approach to an Artificial Intelligence (AI) framework which is inspired from the Yomi concept. The Yomi concept is applied in competitive games where participants have to \textit{read} their opponent's mind.

The proposed AI subsystems in the framework can be categorized into two components: a problem analysis component and a decision making component. Two predictors have been created for the problem analysis component: the Modified Beat Frequent Pick (MBFP) and the Historical Sequence-based Predictor (HSP). The decision making component is comprised of the Yomi AI.

The objectives of the study are as follows:

\begin{enumerate}[label=\arabic*.]

\item How can the Yomi concept be modeled in decision making for Artificial Intelligence?

\item Can the Yomi Estimator help improve the performance of some specific RoShamBo predictors?

\item What parameters can be considered for the Yomi framework?

\item How can these parameters be automatically fine-tuned?

\end{enumerate}

\hspace{1em}

\item \begin{center}\textbf{Findings}\end{center}

\ifnotUCFormat
\else
	\hspace{1em}
\fi

Based on the result of the study, the following findings were drawn:

\begin{enumerate}[label=\arabic*.]
\item The first International RoShamBo Tournament test suite is used to determine the effectiveness of the AI. It used two ranking systems: the number of matches won \textit{(match results)} and the number of turns won \textit{(tournament results)}. The lower the rank, the better it performs.

For RoShamBo Orthogaming, the highest match ranking Yomi AI variant has a ranking of 7; the highest tournament ranking is 13.

For RoShamBo Metagaming, The highest match ranking Yomi AI variants have a ranking of 3; the highest tournament ranking variant is 6.

\item A two-tailed paired t-test was used to check the effectiveness of adding Yomi to each predictor variant. 

\item The different parameters were identified at chapter \ref{chapter:Results} section \ref{ssec:YomiTrainingProgram}.

\item The use of Genetic Algorithm in the context of metagaming allows for the parameters to be fine-tuned.

\end{enumerate}

\hspace{1em}

\item \begin{center}\textbf{Conclusions}\end{center}

\ifnotUCFormat
\else
	\hspace{1em}
\fi

The Yomi AI framework has potential as a competitive AI in the context of RoShamBo. 

\begin{enumerate}[label=\arabic*.]
\item The Yomi Estimator can be used to model the Yomi concept. To allow for the estimator to apply Yomi to AI, the Yomi AI Framework was developed.

\item It was shown that there is significant improvement between the Yomi variants and those without Yomi.

\item A training program was developed that allows for the researcher to develop AI with different parameters for the purpose of research.

\item Metagaming can be used to fine-tune AI parameters.
\end{enumerate}

\hspace{1em}

\item \begin{center}\textbf{Recommendations}\end{center}

\ifnotUCFormat
\else
	\hspace{1em}
\fi

Based on the findings and conclusions, the following are the recommended contributions to Yomi AI research:

\begin{enumerate}[label=\arabic*.]
\item This dissertation focused on developing a general and formal definition of an AI framework using the concept of Yomi. Using this framework, different AIs can be developed.
\item The Yomi concept is applied in competitive games where participants have to \textit{read} their opponent's mind. In this dissertation, the researcher has developed a Yomi estimator based on Markov Chains.
\item This dissertation describes a Genetic Algorithm trainer for use in applying metagaming to the Yomi Estimator and Predictors. This allow the Yomi AI framework to be modified within the context of RoShamBo.
\item This dissertation contributes to the field of Artificial Intelligence and RoShamBo research. The output of this research is a new AI framework that can be used as a basis for new RoShamBo AIs. 
\item RoShamBo is a simple competitive game but future studies can explore the applicability of the Yomi AI framework on other competitive games.
\item The concept of Yomi has been used in competitive games and is a critical aspect of play. At the time of writing, this researcher has not found published research incorporating Yomi into Artificial Intelligence. It is the hope of this researcher that this dissertation will open new doors to utilizing Yomi in AI research.
\end{enumerate}

\end{enumerate}
}
\end{enumerate}

%\vfill

% Note: Above should not exceed three pages in length (coupon bond 8 1/2 by 11")
